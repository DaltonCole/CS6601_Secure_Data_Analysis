\documentclass[times]{article}

\usepackage[margin=1.0in]{geometry}
\usepackage{graphicx}
\usepackage{adjustbox}
\usepackage{float}
\usepackage{placeins}
\usepackage[none]{hyphenat}
\usepackage{amsmath}
\usepackage[us]{datetime}
\usepackage[explicit]{titlesec}
\titleformat{\section}{\normalfont}{}{0em}{\textbf{\large Problem  \thesection}\  \normalsize #1}
\begin{document}
	\title{CS 6601 Secure Data Analysis - Fall 2017 \\ Homework 1}
	\author{Adam Bowers \\ Sammie Bush \\ Dalton Cole}
	\date{\formatdate{22}{9}{2017}}
	\maketitle

	% Problem 1
	\section{Show how to construct the garbled circuit.}
	The normal circuit for constructing an "=" operator can be seen in Figure \ref{fig:normal_circuit}. A circuit labeled with wires can be seen in \ref{fig:garbled_circuit}. To construct a garbled circuit, first each wire has to be assigned two random t bit strings. There are two strings to cover the 0 and 1 input cases. In an actual garbled circuit, t would be set to 80. For demonstrative purposes, t will be 8 in this case. Table \ref{tab:random_string} shows the random strings assigned to each wire where $v_a^b$ such that a is the wire and b is the bit the string represents. Next, a permutation bit must be randomly chosen for each wire. This can be found in Table \ref{tab:permutation}. The random permutation is appended to $v_i^j$ to form $w_i^j$ such that $w_k^0 = v_k^0 \Vert (0 \oplus p_k)$ and $w_k^1 = v_k^1 || (1 \oplus p_k)$. Each $w_i^j$ value can be seen in Table \ref{tab:wire}. 

	Following this, each truth table is replaced with it's corresponding Garbled-Truth-Table (GTT) by replacing each 0 or 1 with $w_k^0$ or $w_k^1$ respectively. The GTT is replaced by the Encrypted-Garbled-Truth-Table (EGTT) which is an encrypted version of the GTT. The encryption is performed by SHA1 hashing $v_i^x \Vert k \Vert x' \Vert y'$ with the plane text $w_k^x$ where $x' = x \oplus p_i$ and  $y' = y \oplus p_j$ and x, y are the entries in the original truth table. This encryption will be represented as $Enc(w_i^j)$. 

	To make the Permutted-Encrypted-Garbled-Truth-Table (PEGTT), the rows in the EGTT have to be swapped based on the following rule: if $p_i = 1$, the first two entries of the table are swapped with the last two entries; if $p_j = 1$, then the first and third are swapped and the second and fourth entries are swapped. The GTT, EGTT, and PEGTT for gate 0 can be found in Table \ref{tab:gtt0}. For gates 1-6, see Tables \ref{tab:gtt1}, \ref{tab:gtt2}, \ref{tab:gtt3}, \ref{tab:gtt4}, \ref{tab:gtt5}, \ref{tab:gtt6}. For another example, see attached.
	

	\begin{figure}
		\caption{Normal Circuit}
		\label{fig:normal_circuit}
		\includegraphics[width=\textwidth]{images/equal_circuit.pdf}
	\end{figure}

	\begin{figure}
		\caption{Garbled Circuit Wires}
		\label{fig:garbled_circuit}
		\includegraphics[width=\textwidth]{images/garbled_circuit.pdf}
	\end{figure}

	\begin{table}
		\centering
		\caption{Random t-bit Strings}
		\label{tab:random_string}
		\begin{tabular}{| c | c |}
			\hline
			$v_0^0$ = 00000000 & $v_0^1$ = 11111111 \\
			\hline
			$v_1^0$ = 00000001 & $v_1^1$ = 10000000 \\
			\hline
			$v_2^0$ = 00000010 & $v_2^1$ = 01000000 \\
			\hline
			$v_3^0$ = 00000011 & $v_3^1$ = 11000000 \\
			\hline
			$v_4^0$ = 00000100 & $v_4^1$ = 00100000 \\
			\hline
			$v_5^0$ = 00000101 & $v_5^1$ = 10100000 \\
			\hline
			$v_6^0$ = 00000110 & $v_6^1$ = 01100000 \\
			\hline
			$v_7^0$ = 00000111 & $v_7^1$ = 11100000 \\
			\hline
			$v_8^0$ = 00001000 & $v_8^1$ = 00010000 \\
			\hline
			$v_9^0$ = 00001001 & $v_9^1$ = 10010000 \\
			\hline
			$v_{10}^0$ = 00001010 & $v_{10}^1$ = 01010000 \\
			\hline
			$v_{11}^0$ = 00001011 & $v_{11}^1$ = 11010000 \\
			\hline
			$v_{12}^0$ = 00001100 & $v_{12}^1$ = 00110000 \\
			\hline
			$v_{13}^0$ = 00001101 & $v_{13}^1$ = 10110000 \\
			\hline
			$v_{14}^0$ = 00001110 & $v_{14}^1$ = 01110000 \\
			\hline
		\end{tabular}
	\end{table}

	\begin{table}
		\centering
		\caption{Permutation Bit}
		\label{tab:permutation}
		\begin{tabular}{| c |}
			\hline
			$p_{0}$ = 0 \\
			\hline
			$p_{1}$ = 1 \\
			\hline
			$p_{2}$ = 0 \\
			\hline
			$p_{3}$ = 1 \\
			\hline
			$p_{4}$ = 0 \\
			\hline
			$p_{5}$ = 1 \\
			\hline
			$p_{6}$ = 0 \\
			\hline
			$p_{7}$ = 1 \\
			\hline
			$p_{8}$ = 0 \\
			\hline
			$p_{9}$ = 1 \\
			\hline
			$p_{10}$ = 0 \\
			\hline
			$p_{11}$ = 1 \\
			\hline
			$p_{12}$ = 0 \\
			\hline
			$p_{13}$ = 1 \\
			\hline
			$p_{14}$ = 0 \\
			\hline
		\end{tabular}
	\end{table}

	\begin{table}
		\centering
		\caption{Wire Strings}
		\label{tab:wire}
		\begin{tabular}{| c | c |}
			\hline
			$w_0^0$ = 000000000 & $w_0^1$ = 111111111 \\
			\hline
			$w_1^0$ = 000000011 & $w_1^1$ = 100000000 \\
			\hline
			$w_2^0$ = 000000100 & $w_2^1$ = 010000001 \\
			\hline
			$w_3^0$ = 000000111 & $w_3^1$ = 110000000 \\
			\hline
			$w_4^0$ = 000001000 & $w_4^1$ = 001000001 \\
			\hline
			$w_5^0$ = 000001011 & $w_5^1$ = 101000000 \\
			\hline
			$w_6^0$ = 000001100 & $w_6^1$ = 011000001 \\
			\hline
			$w_7^0$ = 000001111 & $w_7^1$ = 111000000 \\
			\hline
			$w_8^0$ = 000010000 & $w_8^1$ = 000100001 \\
			\hline
			$w_9^0$ = 000010011 & $w_9^1$ = 100100000 \\
			\hline
			$w_{10}^0$ = 000010100 & $w_{10}^1$ = 010100001 \\
			\hline
			$w_{11}^0$ = 000010111 & $w_{11}^1$ = 110100000 \\
			\hline
			$w_{12}^0$ = 000011000 & $w_{12}^1$ = 001100001 \\
			\hline
			$w_{13}^0$ = 000011011 & $w_{13}^1$ = 101100000 \\
			\hline
			$w_{14}^0$ = 000011100 & $w_{14}^1$ = 011100001 \\
			\hline
		\end{tabular}
	\end{table}


	\begin{table}
		\centering
		\caption{GTT, EGTT, PEGTT for Gate 0}
		\label{tab:gtt0}
		\begin{adjustbox}{width=1\textwidth}
		\begin{tabular}{|c|c|c||c|c|c||c|c|c||c|c|c|}
			\hline
			\multicolumn{3}{|c||}{Truth Table} 		& 
				\multicolumn{3}{|c||}{GTT}			& 
					\multicolumn{3}{|c||}{EGTT} 		& 
						\multicolumn{3}{|c|}{PEGTT} \\
			\hline
			\hline
			X & Y & Out	& 
				X & Y & Out	& 
					X & Y & Out	& 
						X & Y & Out	\\
			\hline
			0 & 0 & 1 	&
				$w_{0}^0$	& $w_{1}^0$	& $w_{8}^1$	& 
					$Enc(w_{0}^0)$	& $Enc(w_{1}^0)$	& $Enc(w_{8}^1)$ &
						$Enc(w_{0}^0)$	& $Enc(w_{1}^1)$	& $Enc(w_{8}^0)$ \\
			\hline
			0 & 1 & 0 	&
				$w_{0}^0$	& $w_{1}^1$	& $w_{8}^0$	& 
					$Enc(w_{0}^0)$	& $Enc(w_{1}^1)$	& $Enc(w_{8}^0)$ &
						$Enc(w_{0}^0)$	& $Enc(w_{1}^0)$	& $Enc(w_{8}^1)$ \\
			\hline
			1 & 0 & 0 	&
				$w_{0}^1$	& $w_{1}^0$	& $w_{8}^0$	& 
					$Enc(w_{0}^1)$	& $Enc(w_{1}^0)$	& $Enc(w_{8}^0)$ &
						$Enc(w_{0}^1)$	& $Enc(w_{1}^1)$	& $Enc(w_{8}^1)$ \\
			\hline
			1 & 1 & 1 	&
				$w_{0}^1$	& $w_{1}^1$	& $w_{8}^1$	& 
					$Enc(w_{0}^1)$	& $Enc(w_{1}^1)$	& $Enc(w_{8}^1)$ &
						$Enc(w_{0}^1)$	& $Enc(w_{1}^0)$	& $Enc(w_{8}^0)$ \\
			\hline
		\end{tabular}
		\end{adjustbox}
	\end{table}

	\begin{table}
		\centering
		\caption{GTT, EGTT, PEGTT for Gate 1}
		\label{tab:gtt1}
		\begin{adjustbox}{width=1\textwidth}
		\begin{tabular}{|c|c|c||c|c|c||c|c|c||c|c|c|}
			\hline
			\multicolumn{3}{|c||}{Truth Table} 		& 
				\multicolumn{3}{|c||}{GTT}			& 
					\multicolumn{3}{|c||}{EGTT} 		& 
						\multicolumn{3}{|c|}{PEGTT} \\
			\hline
			\hline
			X & Y & Out	& 
				X & Y & Out	& 
					X & Y & Out	& 
						X & Y & Out	\\
			\hline
			0 & 0 & 1 	&
				$w_{2}^0$	& $w_{3}^0$	& $w_{9}^1$	& 
					$Enc(w_{2}^0)$	& $Enc(w_{3}^0)$	& $Enc(w_{9}^1)$ &
						$Enc(w_{2}^0)$	& $Enc(w_{3}^1)$	& $Enc(w_{9}^0)$ \\
			\hline
			0 & 1 & 0 	&
				$w_{2}^0$	& $w_{3}^1$	& $w_{9}^0$	& 
					$Enc(w_{2}^0)$	& $Enc(w_{3}^1)$	& $Enc(w_{9}^0)$ &
						$Enc(w_{2}^0)$	& $Enc(w_{3}^0)$	& $Enc(w_{9}^1)$ \\
			\hline
			1 & 0 & 0 	&
				$w_{2}^1$	& $w_{3}^0$	& $w_{9}^0$	& 
					$Enc(w_{2}^1)$	& $Enc(w_{3}^0)$	& $Enc(w_{9}^0)$ &
						$Enc(w_{2}^1)$	& $Enc(w_{3}^1)$	& $Enc(w_{9}^1)$ \\
			\hline
			1 & 1 & 1 	&
				$w_{2}^1$	& $w_{3}^1$	& $w_{9}^1$	& 
					$Enc(w_{2}^1)$	& $Enc(w_{3}^1)$	& $Enc(w_{9}^1)$ &
						$Enc(w_{2}^1)$	& $Enc(w_{3}^0)$	& $Enc(w_{9}^0)$ \\
			\hline
		\end{tabular}
		\end{adjustbox}
	\end{table}

	\begin{table}
		\centering
		\caption{GTT, EGTT, PEGTT for Gate 2}
		\label{tab:gtt2}
		\begin{adjustbox}{width=1\textwidth}
		\begin{tabular}{|c|c|c||c|c|c||c|c|c||c|c|c|}
			\hline
			\multicolumn{3}{|c||}{Truth Table} 		& 
				\multicolumn{3}{|c||}{GTT}			& 
					\multicolumn{3}{|c||}{EGTT} 		& 
						\multicolumn{3}{|c|}{PEGTT} \\
			\hline
			\hline
			X & Y & Out	& 
				X & Y & Out	& 
					X & Y & Out	& 
						X & Y & Out	\\
			\hline
			0 & 0 & 1 	&
				$w_{4}^0$	& $w_{5}^0$	& $w_{10}^1$	& 
					$Enc(w_{4}^0)$	& $Enc(w_{5}^0)$	& $Enc(w_{10}^1)$ &
						$Enc(w_{4}^0)$	& $Enc(w_{5}^1)$	& $Enc(w_{10}^0)$ \\
			\hline
			0 & 1 & 0 	&
				$w_{4}^0$	& $w_{5}^1$	& $w_{10}^0$	& 
					$Enc(w_{4}^0)$	& $Enc(w_{5}^1)$	& $Enc(w_{10}^0)$ &
						$Enc(w_{4}^0)$	& $Enc(w_{5}^0)$	& $Enc(w_{10}^1)$ \\
			\hline
			1 & 0 & 0 	&
				$w_{4}^1$	& $w_{5}^0$	& $w_{10}^0$	& 
					$Enc(w_{4}^1)$	& $Enc(w_{5}^0)$	& $Enc(w_{10}^0)$ &
						$Enc(w_{4}^1)$	& $Enc(w_{5}^1)$	& $Enc(w_{10}^1)$ \\
			\hline
			1 & 1 & 1 	&
				$w_{4}^1$	& $w_{5}^1$	& $w_{10}^1$	& 
					$Enc(w_{4}^1)$	& $Enc(w_{5}^1)$	& $Enc(w_{10}^1)$ &
						$Enc(w_{4}^1)$	& $Enc(w_{5}^0)$	& $Enc(w_{10}^0)$ \\
			\hline
		\end{tabular}
		\end{adjustbox}
	\end{table}

	\begin{table}
		\centering
		\caption{GTT, EGTT, PEGTT for Gate 3}
		\label{tab:gtt3}
		\begin{adjustbox}{width=1\textwidth}
		\begin{tabular}{|c|c|c||c|c|c||c|c|c||c|c|c|}
			\hline
			\multicolumn{3}{|c||}{Truth Table} 		& 
				\multicolumn{3}{|c||}{GTT}			& 
					\multicolumn{3}{|c||}{EGTT} 		& 
						\multicolumn{3}{|c|}{PEGTT} \\
			\hline
			\hline
			X & Y & Out	& 
				X & Y & Out	& 
					X & Y & Out	& 
						X & Y & Out	\\
			\hline
			0 & 0 & 1 	&
				$w_{6}^0$	& $w_{7}^0$	& $w_{11}^1$	& 
					$Enc(w_{6}^0)$	& $Enc(w_{7}^0)$	& $Enc(w_{11}^1)$ &
						$Enc(w_{6}^0)$	& $Enc(w_{7}^1)$	& $Enc(w_{11}^0)$ \\
			\hline
			0 & 1 & 0 	&
				$w_{6}^0$	& $w_{7}^1$	& $w_{11}^0$	& 
					$Enc(w_{6}^0)$	& $Enc(w_{7}^1)$	& $Enc(w_{11}^0)$ &
						$Enc(w_{6}^0)$	& $Enc(w_{7}^0)$	& $Enc(w_{11}^1)$ \\
			\hline
			1 & 0 & 0 	&
				$w_{6}^1$	& $w_{7}^0$	& $w_{11}^0$	& 
					$Enc(w_{6}^1)$	& $Enc(w_{7}^0)$	& $Enc(w_{11}^0)$ &
						$Enc(w_{6}^1)$	& $Enc(w_{7}^1)$	& $Enc(w_{11}^1)$ \\
			\hline
			1 & 1 & 1 	&
				$w_{6}^1$	& $w_{7}^1$	& $w_{11}^1$	& 
					$Enc(w_{6}^1)$	& $Enc(w_{7}^1)$	& $Enc(w_{11}^1)$ &
						$Enc(w_{6}^1)$	& $Enc(w_{7}^0)$	& $Enc(w_{11}^0)$ \\
			\hline
		\end{tabular}
		\end{adjustbox}
	\end{table}

	\begin{table}
		\centering
		\caption{GTT, EGTT, PEGTT for Gate 4}
		\label{tab:gtt4}
		\begin{adjustbox}{width=1\textwidth}
		\begin{tabular}{|c|c|c||c|c|c||c|c|c||c|c|c|}
			\hline
			\multicolumn{3}{|c||}{Truth Table} 		& 
				\multicolumn{3}{|c||}{GTT}			& 
					\multicolumn{3}{|c||}{EGTT} 		& 
						\multicolumn{3}{|c|}{PEGTT} \\
			\hline
			\hline
			X & Y & Out	& 
				X & Y & Out	& 
					X & Y & Out	& 
						X & Y & Out	\\
			\hline
			0 & 0 & 0 	&
				$w_{8}^0$	& $w_{9}^0$	& $w_{12}^0$	& 
					$Enc(w_{8}^0)$	& $Enc(w_{9}^0)$	& $Enc(w_{12}^0)$ &
						$Enc(w_{8}^0)$	& $Enc(w_{9}^1)$	& $Enc(w_{12}^0)$ \\
			\hline
			0 & 1 & 0 	&
				$w_{8}^0$	& $w_{9}^1$	& $w_{12}^0$	& 
					$Enc(w_{8}^0)$	& $Enc(w_{9}^1)$	& $Enc(w_{12}^0)$ &
						$Enc(w_{8}^0)$	& $Enc(w_{9}^0)$	& $Enc(w_{12}^0)$ \\
			\hline
			1 & 0 & 0 	&
				$w_{8}^1$	& $w_{9}^0$	& $w_{12}^0$	& 
					$Enc(w_{8}^1)$	& $Enc(w_{9}^0)$	& $Enc(w_{12}^0)$ &
						$Enc(w_{8}^1)$	& $Enc(w_{9}^1)$	& $Enc(w_{12}^1)$ \\
			\hline
			1 & 1 & 1 	&
				$w_{8}^1$	& $w_{9}^1$	& $w_{12}^1$	& 
					$Enc(w_{8}^1)$	& $Enc(w_{9}^1)$	& $Enc(w_{12}^1)$ &
						$Enc(w_{8}^1)$	& $Enc(w_{9}^0)$	& $Enc(w_{12}^0)$ \\
			\hline
		\end{tabular}
		\end{adjustbox}
	\end{table}

	\begin{table}
		\centering
		\caption{GTT, EGTT, PEGTT for Gate 5}
		\label{tab:gtt5}
		\begin{adjustbox}{width=1\textwidth}
		\begin{tabular}{|c|c|c||c|c|c||c|c|c||c|c|c|}
			\hline
			\multicolumn{3}{|c||}{Truth Table} 		& 
				\multicolumn{3}{|c||}{GTT}			& 
					\multicolumn{3}{|c||}{EGTT} 		& 
						\multicolumn{3}{|c|}{PEGTT} \\
			\hline
			\hline
			X & Y & Out	& 
				X & Y & Out	& 
					X & Y & Out	& 
						X & Y & Out	\\
			\hline
			0 & 0 & 0 	&
				$w_{10}^0$	& $w_{11}^0$	& $w_{13}^0$	& 
					$Enc(w_{10}^0)$	& $Enc(w_{11}^0)$	& $Enc(w_{13}^0)$ &
						$Enc(w_{10}^0)$	& $Enc(w_{11}^1)$	& $Enc(w_{13}^0)$ \\
			\hline
			0 & 1 & 0 	&
				$w_{10}^0$	& $w_{11}^1$	& $w_{13}^0$	& 
					$Enc(w_{10}^0)$	& $Enc(w_{11}^1)$	& $Enc(w_{13}^0)$ &
						$Enc(w_{10}^0)$	& $Enc(w_{11}^0)$	& $Enc(w_{13}^0)$ \\
			\hline
			1 & 0 & 0 	&
				$w_{10}^1$	& $w_{11}^0$	& $w_{13}^0$	& 
					$Enc(w_{10}^1)$	& $Enc(w_{11}^0)$	& $Enc(w_{13}^0)$ &
						$Enc(w_{10}^1)$	& $Enc(w_{11}^1)$	& $Enc(w_{13}^1)$ \\
			\hline
			1 & 1 & 1 	&
				$w_{10}^1$	& $w_{11}^1$	& $w_{13}^1$	& 
					$Enc(w_{10}^1)$	& $Enc(w_{11}^1)$	& $Enc(w_{13}^1)$ &
						$Enc(w_{10}^1)$	& $Enc(w_{11}^0)$	& $Enc(w_{13}^0)$ \\
			\hline
		\end{tabular}
		\end{adjustbox}
	\end{table}

	\begin{table}
		\centering
		\caption{GTT, EGTT, PEGTT for Gate 6}
		\label{tab:gtt6}
		\begin{adjustbox}{width=1\textwidth}
		\begin{tabular}{|c|c|c||c|c|c||c|c|c||c|c|c|}
			\hline
			\multicolumn{3}{|c||}{Truth Table} 		& 
				\multicolumn{3}{|c||}{GTT}			& 
					\multicolumn{3}{|c||}{EGTT} 		& 
						\multicolumn{3}{|c|}{PEGTT} \\
			\hline
			\hline
			X & Y & Out	& 
				X & Y & Out	& 
					X & Y & Out	& 
						X & Y & Out	\\
			\hline
			0 & 0 & 0 	&
				$w_{12}^0$	& $w_{13}^0$	& $w_{14}^0$	& 
					$Enc(w_{12}^0)$	& $Enc(w_{13}^0)$	& $Enc(w_{14}^0)$ &
						$Enc(w_{12}^0)$	& $Enc(w_{13}^1)$	& $Enc(w_{14}^0)$ \\
			\hline
			0 & 1 & 0 	&
				$w_{12}^0$	& $w_{13}^1$	& $w_{14}^0$	& 
					$Enc(w_{12}^0)$	& $Enc(w_{13}^1)$	& $Enc(w_{14}^0)$ &
						$Enc(w_{12}^0)$	& $Enc(w_{13}^0)$	& $Enc(w_{14}^0)$ \\
			\hline
			1 & 0 & 0 	&
				$w_{12}^1$	& $w_{13}^0$	& $w_{14}^0$	& 
					$Enc(w_{12}^1)$	& $Enc(w_{13}^0)$	& $Enc(w_{14}^0)$ &
						$Enc(w_{12}^1)$	& $Enc(w_{13}^1)$	& $Enc(w_{14}^1)$ \\
			\hline
			1 & 1 & 1 	&
				$w_{12}^1$	& $w_{13}^1$	& $w_{14}^1$	& 
					$Enc(w_{12}^1)$	& $Enc(w_{13}^1)$	& $Enc(w_{14}^1)$ &
						$Enc(w_{12}^1)$	& $Enc(w_{13}^0)$	& $Enc(w_{14}^0)$ \\
			\hline
		\end{tabular}
		\end{adjustbox}
	\end{table}

	\section{Show how to evaluate the garbed circuit securely.}
	To evaluate the garbled circuit, Bob sends to Alice the garbled circuit along with his wired input, in this case $w_0^x$, $w_2^x$, $w_4^x$, and $w_6^x$. For Alice to obtain her garbled input ($w_1^y$, $w_3^y$, $w_5^y$, and $w_7^y$), she must use oblivious transfer (OT) with Bob.

	For each gate, the following must take place. Alice splits $w_i^x$ into $v_i$ and $x$, where $x$ equals the least significant bit of $W_i^x$ and $v_i$ is the remainder of $W_i^x$. Similarly, split $W_j^y$ into $v_j$ and $y$, where $y$ equals the least significant bit of $W_j^y$ and $v_j$ is the remainder of $w_j^y$. The values of $x$ and $y$ are used to index into the PEGTT where $x$ is the high bit and $y$ is the low bit. To retrieve $w_k$, Alice has to hash $v_i \Vert k \Vert x \Vert y$ and $v_j \Vert k \Vert x \Vert y$, and then XOR the two hashes with the ciphertext.

	Alice then uses the transition table she got from Bob to interpret the circuit's true output. She sends the garbled value to Bob, and Bob maps the garbled value to the true value as well.

	For example, let's say Bob has a value of $1011$ and Alice has a value of $1010$. The following tables illistrates how to evaulate the garbled circuit securely: \ref{tab:DG0}, \ref{tab:DG1}, \ref{tab:DG2}, \ref{tab:DG3}, \ref{tab:DG4}, \ref{tab:DG5}, \ref{tab:DG6}.


	\begin{table}
		\centering
		\caption{Decrypting Gate 0}
		\label{tab:DG0}
		\begin{tabular}{|c|c||c|c|}
			\hline
			\multicolumn{2}{|c||}{$w_0$} & \multicolumn{2}{|c|}{$w_1$} \\
			\hline
			$v_0 = 11111111$ & $x = 1$ & $v_1 = 10000000$ & $y = 0$ \\
			\hline
			\multicolumn{4}{|c||}{$w_8 = hash(v_0 \Vert 8 \Vert x \Vert y) \oplus hash(v_1 \Vert 8 \Vert x \Vert y) \oplus Enc(w_8^1)$} \\
			\hline
			\multicolumn{4}{|c||}{Transition Table($w_8^1$) = 1} \\
			\hline
		\end{tabular}
	\end{table}

	\begin{table}
		\centering
		\caption{Decrypting Gate 1}
		\label{tab:DG1}
		\begin{tabular}{|c|c||c|c|}
			\hline
			\multicolumn{2}{|c||}{$w_2$} & \multicolumn{2}{|c|}{$w_3$} \\
			\hline
			$v_2 = 10000000$ & $x = 0$ & $v_3 = 00000011$ & $y = 1$ \\
			\hline
			\multicolumn{4}{|c||}{$w_9 = hash(v_2 \Vert 9 \Vert x \Vert y) \oplus hash(v_3 \Vert 9 \Vert x \Vert y) \oplus Enc(w_9^1)$} \\
			\hline
			\multicolumn{4}{|c||}{Transition Table($w_9^1$) = 1} \\
			\hline
		\end{tabular}
	\end{table}

	\begin{table}
		\centering
		\caption{Decrypting Gate 2}
		\label{tab:DG2}
		\begin{tabular}{|c|c||c|c|}
			\hline
			\multicolumn{2}{|c||}{$w_4$} & \multicolumn{2}{|c|}{$w_5$} \\
			\hline
			$v_4 = 001000001$ & $x = 1$ & $v_5 = 101000000$ & $y = 0$ \\
			\hline
			\multicolumn{4}{|c||}{$w_{10} = hash(v_4 \Vert 10 \Vert x \Vert y) \oplus hash(v_5 \Vert 10 \Vert x \Vert y) \oplus Enc(w_{10}^1)$} \\
			\hline
			\multicolumn{4}{|c||}{Transition Table($w_{10}^1$) = 1} \\
			\hline
		\end{tabular}
	\end{table}

	\begin{table}
		\centering
		\caption{Decrypting Gate 3}
		\label{tab:DG3}
		\begin{tabular}{|c|c||c|c|}
			\hline
			\multicolumn{2}{|c||}{$w_6$} & \multicolumn{2}{|c|}{$w_7$} \\
			\hline
			$v_6 = 01100000$ & $x = 1$ & $v_7 = 11100000$ & $y = 0$ \\
			\hline
			\multicolumn{4}{|c||}{$w_{11} = hash(v_6 \Vert 11 \Vert x \Vert y) \oplus hash(v_7 \Vert 11 \Vert x \Vert y) \oplus Enc(w_{11}^0)$} \\
			\hline
			\multicolumn{4}{|c||}{Transition Table($w_{11}^0$) = 0} \\
			\hline
		\end{tabular}
	\end{table}

	\begin{table}
		\centering
		\caption{Decrypting Gate 4}
		\label{tab:DG4}
		\begin{tabular}{|c|c||c|c|}
			\hline
			\multicolumn{2}{|c||}{$w_8$} & \multicolumn{2}{|c|}{$w_9$} \\
			\hline
			$v_8 = 00010000$ & $x = 1$ & $v_9 = 10010000$ & $y = 0$ \\
			\hline
			\multicolumn{4}{|c||}{$w_{12} = hash(v_8 \Vert 12 \Vert x \Vert y) \oplus hash(v_9 \Vert 12 \Vert x \Vert y) \oplus Enc(w_{12}^0)$} \\
			\hline
			\multicolumn{4}{|c||}{Transition Table($w_{12}^1$) = 1} \\
			\hline
		\end{tabular}
	\end{table}

	\begin{table}
		\centering
		\caption{Decrypting Gate 5}
		\label{tab:DG5}
		\begin{tabular}{|c|c||c|c|}
			\hline
			\multicolumn{2}{|c||}{$w_{10}$} & \multicolumn{2}{|c|}{$w_{11}$} \\
			\hline
			$v_{10} = 01010000$ & $x = 1$ & $v_{11} = 00001011$ & $y = 1$ \\
			\hline
			\multicolumn{4}{|c||}{$w_{13} = hash(v_{10} \Vert 13 \Vert x \Vert y) \oplus hash(v_{11} \Vert 13 \Vert x \Vert y) \oplus Enc(w_{13}^0)$} \\
			\hline
			\multicolumn{4}{|c||}{Transition Table($w_{13}^0$) = 0} \\
			\hline
		\end{tabular}
	\end{table}

	\begin{table}
		\centering
		\caption{Decrypting Gate 6}
		\label{tab:DG6}
		\begin{tabular}{|c|c||c|c|}
			\hline
			\multicolumn{2}{|c||}{$w_{12}$} & \multicolumn{2}{|c|}{$w_{13}$} \\
			\hline
			$v_{12} = 00110000$ & $x = 1$ & $v_{13} = 00001101$ & $y = 1$ \\
			\hline
			\multicolumn{4}{|c||}{$w_{14} = hash(v_{12} \Vert 14 \Vert x \Vert y) \oplus hash(v_{13} \Vert 14 \Vert x \Vert y) \oplus Enc(w_{14}^0)$} \\
			\hline
			\multicolumn{4}{|c||}{Transition Table($w_{14}^0$) = 0} \\
			\hline
		\end{tabular}
	\end{table}
	
	\section{Show how the evaluator knows which entry in the garbled circuit can be decrypted correctly.}
	In a 1-2 OT, Bob has two messages $m_0$ and $m_1$ and Alice has a bit b. Alice wants to retrieve $m_b$, without Bob knowing b.

	\begin{enumerate}
  		\item Bob sends N, e, $x_0$, and $x_1$ to Alice, i.e. $x_0$, $x_1$ are randomly chosen from ${1, ..., N-1}$. (Bob also knows the private key $d$)
  		\item Alice randomly selects $k \in {1,...,N-1}$
  		\item Alice sends $v = (k^e mod N) \oplus x_b$ to Bob
  		\item Bob computes $k_0 = (v \oplus x_0)^d mod N = k$. $k_1 = (v \oplus x_1)^d mod N = k$, i.e. $k^e \oplus x_0 \oplus x_0 \Rightarrow k$ and $k^e \oplus x_1 \oplus x_0 \Rightarrow k’$
  		\item Bob sends  $z_0 = m_0 \oplus k_0 = m_0 \oplus k$ and $z_1 = m_1 \oplus k_1 = m_0 \oplus k’$ to Alice
  		\item Alice computes $m_b = z_b \oplus k$, i.e. Alice is only able to retrieve one value of either $x_0$ or $x_1$ while knowing k

	\end{enumerate}

	\section{What is the appropriate key size for constructing the garbled circuit?}
	“Fairplay - A Secure Two-Party Computation System” used 80 bits for the key size(it means key size as in size of v’s generated right?). However that alone doesn’t mean the key size is appropriate. The key needs to be large enough to firstly, mask a given input to the circuit. Given the examples in the paper only used inputs of 1 bit, this isn’t an issue for those circuits. The other concern when deciding a key size is to make brute force guessing of the keys or other attacks impractical. Now the necessary size to combat this is debatable and always changing with advancements in hardware and theory, but in regards to papers given 80 bits should suffice.
	
	
		
\end{document}